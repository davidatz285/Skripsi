\documentclass[a4paper,twoside]{article}
\usepackage[T1]{fontenc}
\usepackage[bahasa]{babel}
\usepackage{graphicx}
\usepackage{graphics}
\usepackage{float}
\usepackage[cm]{fullpage}
\pagestyle{myheadings}
\usepackage{etoolbox}
\usepackage{setspace} 
\usepackage{lipsum} 
\setlength{\headsep}{30pt}
\usepackage[inner=2cm,outer=2.5cm,top=2.5cm,bottom=2cm]{geometry} %margin
% \pagestyle{empty}

\makeatletter
\renewcommand{\@maketitle} {\begin{center} {\LARGE \textbf{ \textsc{\@title}} \par} \bigskip {\large \textbf{\textsc{\@author}} }\end{center} }
\renewcommand{\thispagestyle}[1]{}
\markright{\textbf{\textsc{Laporan Perkembangan Pengerjaan Skripsi\textemdash Sem. Genap 2015/2016}}}

\onehalfspacing
 
\begin{document}

\title{\@judultopik}
\author{\nama \textendash \@npm} 

%ISILAH DATA BERIKUT INI:
\newcommand{\nama}{David Christopher Sentosa}
\newcommand{\@npm}{2017730015}
\newcommand{\tanggal}{14/01/2021} %Tanggal pembuatan dokumen
\newcommand{\@judultopik}{Pembangunan Aplikasi \textit{Android} IFStudentPortal} % Judul/topik anda
\newcommand{\kodetopik}{PAN4904}
\newcommand{\jumpemb}{1} % Jumlah pembimbing, 1 atau 2
\newcommand{\pembA}{Pascal Alfadian Nugroho}
\newcommand{\pembB}{-}
\newcommand{\semesterPertama}{49 - Ganjil 20/21} % semester pertama kali topik diambil, angka 1 dimulai dari sem Ganjil 96/97
\newcommand{\lamaSkripsi}{1} % Jumlah semester untuk mengerjakan skripsi s.d. dokumen ini dibuat
\newcommand{\kulPertama}{Skripsi 1} % Kuliah dimana topik ini diambil pertama kali
\newcommand{\tipePR}{B} % tipe progress report :
% A : dokumen pendukung untuk pengambilan ke-2 di Skripsi 1
% B : dokumen untuk reviewer pada presentasi dan review Skripsi 1
% C : dokumen pendukung untuk pengambilan ke-2 di Skripsi 2

% Dokumen hasil template ini harus dicetak bolak-balik !!!!

\maketitle

\pagenumbering{arabic}

\section{Data Skripsi} %TIDAK PERLU MENGUBAH BAGIAN INI !!!
Pembimbing utama/tunggal: {\bf \pembA}\\
Pembimbing pendamping: {\bf \pembB}\\
Kode Topik : {\bf \kodetopik}\\
Topik ini sudah dikerjakan selama : {\bf \lamaSkripsi} semester\\
Pengambilan pertama kali topik ini pada : Semester {\bf \semesterPertama} \\
Pengambilan pertama kali topik ini di kuliah : {\bf \kulPertama} \\
Tipe Laporan : {\bf \tipePR} -
\ifdefstring{\tipePR}{A}{
			Dokumen pendukung untuk {\BF pengambilan ke-2 di Skripsi 1} }
		{
		\ifdefstring{\tipePR}{B} {
				Dokumen untuk reviewer pada presentasi dan {\bf review Skripsi 1}}
			{	Dokumen pendukung untuk {\bf pengambilan ke-2 di Skripsi 2}}
		}
		
\section{Latar Belakang}
Portal Akademik Mahasiswa adalah sistem informasi berbasis \textit{web} yang digunakan oleh mahasiswa UNPAR. Portal Akademik Mahasiswa berfungsi menyediakan informasi akademis mahasiswa. IFStudentPortal adalah sistem informasi berbasis \textit{web} khusus mahasiswa Teknik Informatika UNPAR hasil pengembangan yang lebih lanjut dari Student Portal UNPAR sebagai skripsi milik Herfan Heryandi, dan dikembangkan lebih lanjut untuk mendukung kurikulum 2018 di skripsi milik Andrianto Sugiarto. IFStudentPortal mengambil dan menggunakan data Portal Akademik Mahasiswa. IFStudentPortal dibuat untuk menyediakan fitur-fitur yang belum dimiliki Portal Akademik Mahasiswa pada tahun 2015. IFStudentPortal memiliki 3 fitur utama yaitu :
\begin{enumerate}
    \item Melihat jadwal kuliah yang berbentuk grafik dan terurut berdasarkan hari.
    \item Melihat ringkasan data akademdik. 
    \item Melihat rincian pemenuhan syarat kelulusan.
\end{enumerate}
Skripsi ini dibuat untuk membawa IFStudentPortal ke platform \textit{android} sebagai aplikasi \textit{native android} IFStudentPortal. Aplikasi \textit{android} IFStudentPortal akan dibangun dengan bantuan \textit{library} jsoup dan SIA Models, dan mengikuti panduan \textit{Android Design} untuk memberikan antarmuka pengguna dan pengalaman pengguna aplikasi yang konsisten sesuai standar \textit{Android}.
\section{Rumusan Masalah}
Rumusan masalah yang akan dibahas di skripsi ini adalah sebagai berikut :
\begin{itemize}
    \item Fitur apa saja yang akan tersedia di aplikasi \textit{android} IFStudentPortal?
    \item Bagaimana cara mengimplementasikan fungsi aplikasi \textit{android} IFStudentPortal dengan \textit{library} jsoup?
    \item Bagaimana cara mengimplementasikan tampilan aplikasi \textit{android} IFStudentPortal dengan mengikuti pedoman \textit{Android Design}?
\end{itemize}
\section{Tujuan}
Tujuan yang ingin dicapai dari penulisan skripsi ini sebagai berikut :
\begin{itemize}
    \item Mengetahui fitur apa saja yang akan tersedia di aplikasi \textit{android} IFStudentPortal.
    \item Mengimplementasikan fungsional aplikasi \textit{android} IFStudentPortal menggunakan \textit{library} jsoup.
    \item Mengimplementasikan tampilan aplikasi \textit{android} IFStudentPortal mengikuti pedoman \textit{android design}.
\end{itemize}

\section{Detail Perkembangan Pengerjaan Skripsi}
Detail bagian pekerjaan skripsi sesuai dengan rencana kerja/laporan perkembangan terkahir :
	\begin{enumerate}
		\item \textbf{Melakukan studi mengenai \textit{library} jsoup.}\\
		{\bf Status :} Ada sejak rencana kerja skripsi.\\
		{\bf Hasil :} \textit{Library} jsoup sudah dipelajari secara mendalam untuk digunakan dalam implementasi fitur-fitur aplikasi \textit{android} IFStudentPortal. Kelas-kelas yang dipakai dari \textit{library} jsoup ini sudah didokumentasikan dengan lengkap di Bab 2 Dasar Teori.
		
		\item \textbf{Melakukan studi mengenai pedoman \textit{Android Design}.}\\
		{\bf Status :} Ada sejak rencana kerja skripsi.\\
		{\bf Hasil :} Pedoman \textit{Android Design} sudah dipelajari dan didokumentasikan di Bab 2, dan dilakukan analisis untuk memenuhi pedoman \textit{Android Design} di Bab 3. Aplikasi \textit{Android} IFStudentPortal sudah mulai diimplementasi dengan mengikuti pedoman \textit{Android Design}. 

		\item \textbf{Menganalisis IFStudentPortal, Portal Akademik Mahasiswa, dan kebutuhan aplikasi.}\\
		{\bf Status :} Ada sejak rencana kerja skripsi.\\
		{\bf Hasil :} Portal Akademik Mahasiswa sudah didokumentasikan di Bab 2. IFStudentPortal sudah dianalisis di Bab 3. Saat ini fokus utamanya adalah membuat semua fitur yang ada di situs IFStudentPortal ada di aplikasi \textit{Android} IFStudentPortal, sehingga untuk analisis kebutuhan tambahan akan dilakukan setelah fitur utama selesai.

		\item \textbf{Mendesain kelas aplikasi.}\\
		{\bf Status :} Ada sejak rencana kerja skripsi.\\
		{\bf Hasil :} Perancangan kelas-kelas sudah dilakukan karena aplikasi sudah mulai dibuat, namun perancangannya masih bisa berubah dan belum ditulis karena belum menulis Bab 4.

		\item \textbf{Membangun aplikasi \textit{android} IFStudentPortal dengan \textit{library} jsoup.}\\
		{\bf Status :} Ada sejak rencana kerja skripsi.\\
		{\bf Hasil :} Dari 3 fitur utama yang dijelaskan di bagian latar belakang, saat ini fitur jadwal sudah selesai dan sedang mengerjakan fitur nilai. Saat ini fokus utamanya adalah membuat semua fitur yang ada di situs IFStudentPortal ada di aplikasi \textit{Android} IFStudentPortal, baru menambahkan fitur yang bisa diintegrasikan dengan platform \textit{Android}.

		\item \textbf{Mengimplementasikan tampilan aplikasi sesuai pedoman \textit{Android Design} dan senada terhadap situs IFStudentPortal.}\\
		{\bf Status :} Ada sejak rencana kerja skripsi. \\
		{\bf Hasil :} Bagian tampilan aplikasi yang sudah ada diimplementasi mengikuti pedoman dari \textit{Android Design} dan pemenuhan pedomannya ditulis di Bab 3.

		\item \textbf{Melakukan pengujian dan eksperimen} \\
		{\bf Status :} Ada sejak rencana kerja skripsi.\\
		{\bf Hasil :} Pengujian dilakukan sejalan dengan pembuatan aplikasi, namun eksperimen belum dilakukan.

		\item \textbf{Menulis dokumen skripsi}\\
		{\bf Status :} Ada sejak rencana kerja skripsi.\\
		{\bf Hasil :} Bab 1, Bab 2, Bab 3 sudah selesai ditulis, namun tidak menutup kemungkinan untuk terjadi perubahan.
		
		\item \textbf{Melakukan perawatan terhadap situs IFStudentPortal dan \textit{library} SIA Models}\\
		{\bf Status :} baru ditambahkan pada semester ini\\
		{\bf Hasil :} memperbaiki \textit{bug} yang ada di IFStudentPortal dan sedikit memodifikasi SIA Models agar bisa digunakan di platform \textit{Android}. 
	\end{enumerate}

\section{Pencapaian Rencana Kerja}
Langkah-langkah kerja yang berhasil diselesaikan dalam Skripsi 1 ini adalah sebagai berikut:
\begin{enumerate}
\item Melakukan studi mengenai \textit{library} jsoup.
\item Melakukan studi mengenai pedoman \textit{Android Design}.
\item Menganalisis IFStudentPortal, Portal Akademik Mahasiswa, dan kebutuhan aplikasi.
\item Menulis dokumen skripsi Bab 1-3.
\end{enumerate}



% \section{Kendala yang Dihadapi}
% %TULISKAN BAGIAN INI JIKA DOKUMEN ANDA TIPE A ATAU C
% Kendala - kendala yang dihadapi selama mengerjakan skripsi :
% \begin{itemize}
% 	\item Terlalu banyak melakukan prokratinasi
% 	\item Terlalu banyak godaan berupa hiburan (game, film, dll)
% 	\item Skripsi diambil bersamaan dengan kuliah ASD karena selama 5 semester pertama kuliah tersebut sangat dihindari dan tidak diambil, dan selama 4 semester terakhir kuliah tersebut selalu mendapat nilai E
% 	\item Mengalami kesulitan pada saat sudah mulai membuat program komputer karena selama ini selalu dibantu teman
% \end{itemize}

\vspace{1cm}
\centering Bandung, \tanggal\\
\vspace{2cm} \nama \\ 
\vspace{1cm}

Menyetujui, \\
\ifdefstring{\jumpemb}{2}{
\vspace{1.5cm}
\begin{centering} Menyetujui,\\ \end{centering} \vspace{0.75cm}
\begin{minipage}[b]{0.45\linewidth}
% \centering Bandung, \makebox[0.5cm]{\hrulefill}/\makebox[0.5cm]{\hrulefill}/2013 \\
\vspace{2cm} Nama: \pembA \\ Pembimbing Utama
\end{minipage} \hspace{0.5cm}
\begin{minipage}[b]{0.45\linewidth}
% \centering Bandung, \makebox[0.5cm]{\hrulefill}/\makebox[0.5cm]{\hrulefill}/2013\\
\vspace{2cm} Nama: \pembB \\ Pembimbing Pendamping
\end{minipage}
\vspace{0.5cm}
}{
% \centering Bandung, \makebox[0.5cm]{\hrulefill}/\makebox[0.5cm]{\hrulefill}/2013\\
\vspace{2cm} Nama: \pembA \\ Pembimbing Tunggal
}
\end{document}

