%versi 2 (8-10-2016) 
\chapter{Pendahuluan}
\label{chap:intro}
   
\section{Latar Belakang}
\label{sec:label}

Portal Akademik Mahasiswa adalah sistem informasi berbasis \textit{web} yang digunakan oleh mahasiswa UNPAR \cite{studentportalunpar}. IFStudentPortal adalah sistem informasi berbasis \textit{web} khusus mahasiswa Teknik Informatika UNPAR hasil pengembangan  lebih lanjut dari Portal Akademik Mahasiswa \cite{herfan:15:portal}. 

Skripsi ini dibuat untuk membawa IFStudentPortal ke platform \textit{android} sebagai aplikasi \textit{native android} IFStudentPortal dengan dibantu \textit{library} jsoup dan mengikuti pedoman \textit{android design}. 

 Aplikasi akan dibangun dengan bahasa \textit{native android} yaitu Java. Fitur-fitur yang tersedia di situs IFStudentPortal akan dibawa ke platform \textit{android} dan dibuat agar terintegrasi dengan fitur yang dimiliki \textit{android.} Aplikasi \textit{android} IFStudentPortal akan mengambil data langsung dari Portal Akademik Mahasiswa dengan dibantu \textit{library} jsoup. Untuk melakukan pengambilan data, jsoup akan langsung berkomunikasi dengan Portal Akademik Mahasiswa.  

\section{Rumusan Masalah}
\label{sec:rumusan}
Rumusan masalah yang akan dibahas di skripsi ini adalah sebagai berikut :
\begin{itemize}
    \item Fitur apa saja yang akan tersedia di aplikasi \textit{android} IFStudentPortal?
    \item Bagaimana cara mengimplementasikan fungsi aplikasi \textit{android} IFStudentPortal dengan \textit{library} jsoup?
    \item Bagaimana cara mengimplementasikan tampilan aplikasi \textit{android} IFStudentPortal dengan mengikuti pedoman \textit{android design}?
\end{itemize}

\section{Tujuan}
\label{sec:tujuan}
Tujuan yang ingin dicapai dari penulisan skripsi ini sebagai berikut :
\begin{itemize}
    \item Mengetahui fitur apa saja yang akan tersedia di IFStudentPortal.
    \item Mengimplementasikan fungsional aplikasi \textit{android} IFStudentPortal menggunakan \textit{library} jsoup..
    \item Mengimplementasikan tampilan aplikasi \textit{android} IFStudentPortal mengikuti pedoman \textit{android design}.
    
\end{itemize}

\section{Batasan Masalah}
\label{sec:batasan}
Beberapa batasan yang dibuat terkait dengan pengerjaan skripsi ini adalah sebagai berikut:
\begin{enumerate}
	\item Aplikasi akan diuji menggunakan format NPM mahasiswa angkatan 2017 ke atas.
	\item Aplikasi hanya akan diuji pada perangkat jenis ponsel pintar.
    \item Panduan \textit{Material Design} difokuskan pada panduan aksesibilitas dan panduan platform.
\end{enumerate}

\section{Metodologi}
\label{sec:metlit}
Metodologi yang dilakukan pada skripsi ini adalah sebagai berikut:

\begin{enumerate}
	\item Melakukan studi mengenai \textit{library} jsoup untuk mengambil data dari Portal Akademik Mahasiswa, \textit{android design}, dan skripsi Herfan Heryandi untuk membangun aplikasi \textit{android} IFStudentPortal.
	\item Menganalisis Portal Akademik Mahasiswa dan IFStudentPortal.
	\item Merancang model IFStudentPortal.
	\item Mengimplementasi aplikasi \textit{android} IFStudentPortal menggunakan \textit{library} jsoup dengan mengikuti pedoman \textit{android design}.
	\item Melakukan eksperimen dan pengujian.
	\item Membuat dokumentasi.
\end{enumerate}



\section{Sistematika Pembahasan}
\label{sec:sispem}
Sistematika penulisan setiap bab pada skripsi ini adalah sebagai berikut:

\begin{enumerate}
  \item Bab Pendahuluan \\
  Bab 1 berisi latar belakang, rumusan masalah, tujuan, metode penelitian,
  dan sistematika penulisan yang digunakan untuk menyusun skripsi ini.
  \item Bab Dasar Teori \\
  Bab 2 berisi teori-teori yang digunakan dalam pembuatan skripsi ini. Teori
  yang digunakan yaitu \textit{library} jsoup, SIA Models, dan \textit{android design}.
  \item Bab Analisis \\
  Bab 3 berisi analisis yang dilakukan pada skripsi ini, meliputi analisis sistem, analisis kebutuhan aplikasi \textit{android} IFStudentPortal, analisis pedoman \textit{android design}, analisis Portal Akademik Mahasiswa dan situs IFStudentPortal untuk fitur IFStudentPortal, analisis \textit{use case} meliputi diagram \textsl{use case} dan skenario, serta analisis desain kelas. 
  \item Bab Perancangan \\
  Bab 4 berisi perancangan aplikasi, meliputi diagram kelas rinci beserta deskripsi kelas dan fungsinya dan perancangan antarmuka aplikasi.   
  \item Bab Implementasi dan Pengujian \\
  Bab 5 berisi implementasi dan pengujian aplikasi, meliputi lingkungan implementasi, hasil implementasi, pengujian fungsional, dan pengujian eksperimental.
  \item Bab Kesimpulan dan Saran \\
  Bab 6 berisi kesimpulan dari hasil pembangunan aplikasi beserta saran untuk pengembangan berikutnya.
\end{enumerate}
