%versi 2 (8-10-2016) 
\chapter{Pendahuluan}
\label{chap:intro}
   
\section{Latar Belakang}
\label{sec:label}

Student Portal UNPAR adalah sistem informasi berbasis \textit{web} yang digunakan oleh mahasiswa UNPAR. IF Student Portal adalah sistem informasi berbasis \textit{web} khusus mahasiswa Teknik Informatika UNPAR hasil pengembangan yang lebih lanjut dari Student Portal UNPAR. Skripsi ini dibuat untuk membawa IF Student Portal ke platform \textit{android} sebagai aplikasi \textit{native android} IF Student Portal dengan menggunakan \textit{library} jsoup untuk mengimplementasikan web scraping. 

\textit{Web scraping} adalah teknik untuk mendapatkan informasi tertentu dengan cara yang semi terstruktur dari sebuah \textit{website}. \textit{Web scraping} bisa digunakan ketika \textit{website} yang memiliki informasi informasi yang kita cari tidak menyediakan \textit{API (Application Programming Interface)}. \textit{Library} jsoup\cite{jsoup} merupakan \textit{library} Java yang digunakan untuk melakukan \textit{web scraping} tersebut. Aplikasi akan menggunakan arsitektur \textit{Model-View-Presenter}. Aplikasi akan dibangun dengan bahasa \textit{native android} yaitu Java.

\dtext{5-10}

\section{Rumusan Masalah}
\label{sec:rumusan}
Rumusan masalah yang akan dibahas di skripsi ini adalah sebagai berikut :
\begin{itemize}
    \item Fitur apa saja yang akan tersedia di aplikasi \textit{android} IF Student Portal?
    \item Bagaimana cara mengimplementasikan \textit{web scraping} dengan \textit{library} jsoup?
    \item Bagaimana membangun aplikasi \textit{android} IF Student Portal?
\end{itemize}
\dtext{6}

\section{Tujuan}
\label{sec:tujuan}
Tujuan yang ingin dicapai dari penulisan skripsi ini sebagai berikut :
\begin{itemize}
    \item Mengetahui fitur apa saja yang akan tersedia di IF Student Portal.
    \item Mengimplementasikan \textit{web scraping} menggunakan \textit{library} jsoup.
    \item Membangun aplikasi \textit{native android} IF Student Portal.
    
\end{itemize}

\section{Batasan Masalah}
\label{sec:batasan}
Beberapa batasan yang dibuat terkait dengan pengerjaan skripsi ini adalah sebagai berikut:
\begin{enumerate}
	\item Aplikasi hanya akan menangani format NPM mahasiswa angkatan 2018 ke atas.
	\item Aplikasi hanya akan dikembangkan untuk perangkat dengan sistem operasi \textit{android}.
	\item Mata kuliah yang tersedia hanya mata kuliah yang didukung SIA Models.
	\item Aplikasi tidak dapat digunakan mahasiswa yang diluar IF UNPAR.
\end{enumerate}

\section{Metodologi}
\label{sec:metlit}
Metodologi yang dilakukan pada skripsi ini adalah sebagai berikut:

\begin{enumerate}
	\item Melakukan studi mengenai \textit{library} jsoup untuk mengambil data dari Portal Akademik Mahasiswa, CSS Selector yang akan digunakan jsoup untuk menyeleksi data yang akan diambil, skripsi Herfan Heryandi, untuk membangun Informatika Student Portal.
	\item Menganalisis Student Portal UNPAR dan IF Student Portal.
	\item Merancang model IF Student Portal.
	\item Mengimplementasikan ekstraksi data situs web menggunakan \textit{library} jsoup.
	\item Melakukan eksperimen dan pengujian.
	\item Membuat dokumentasi.
\end{enumerate}



\section{Sistematika Pembahasan}
\label{sec:sispem}
Sistematika penulisan setiap bab pada skripsi ini adalah sebagai berikut:

\begin{enumerate}
  \item Bab Pendahuluan \\
  Bab 1 berisi latar belakang, rumusan masalah, tujuan, metode penelitian,
  dan sistematika penulisan yang digunakan untuk menyusun skripsi ini.
  \item Bab Dasar Teori \\
  Bab 2 berisikan teori-teori yang digunakan dalam pembuatan skripsi ini. Teori
  yang digunakan yaitu \textit{library} jsoup, SIA Models, dan \textit{android design}.
  \item Bab Analisis \\
  Bab 3 berisikan analisis yang dilakukan pada skripsi ini, meliputi analisis sistem, analisis kebutuhan \textit{android} IF Student Portal, analisis komunikasi Portal Akademik Mahasiswa untuk fitur IF Student Portal, analisis \textit{use case} meliputi diagram \textsl{use case} dan skenario, serta analisis kelas. 
	\item Bab Perancangan \\
  Bab 4 berisikan perancangan aplikasi, meliputi diagram kelas rinci beserta deskripsi kelas dan fungsinya dan perancangan antarmuka aplikasi.   \item Bab Implementasi dan Pengujian \\
  Bab 5 berisikan implementasi dan pengujian aplikasi, meliputi lingkungan implementasi, hasil implementasi, pengujian fungsional, dan pengujian eksperimental.
	\item Bab Kesimpulan dan Saran \\
  Bab 6 berisikan kesimpulan dari hasil pembangunan aplikasi beserta saran untuk pengembangan berikutnya.
\end{enumerate}
