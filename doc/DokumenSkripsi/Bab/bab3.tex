\chapter{Analisis}
\label{chap:analisis}

Pada bab ini akan dijelaskan analisis sistem yang sudah berjalan dan sistem yang akan dibangun. Analisis yang akan dibahas meliputi analisis 

\section{Analisis Sistem Berjalan}
%Analisis student portal unpar dan if student portal
Student Portal UNPAR adalah situs yang diperuntukan bagi mahasiswa UNPAR untuk mengakses informasi yang berhubungan dengan studi dan mahasiswa pemilik akun\cite{BTI:2012}. Mahasiswa dapat mengakses Student Portal UNPAR melalui URL \url{https://studentportal.unpar.ac.id/}. Untuk mengakses Portal Akademik Mahasiswa, mahasiswa harus \textit{login} menggunakan akun email \textit{student}. 

IF Student Portal adalah situs yang diperuntukan bagi mahasiswa IF UNPAR\cite{ifstudentportalunpar}. IF Student Portal dibuat dengan tujuan menjadi pengembangan lebih lanjut dari Student Portal UNPAR pada masanya. IF Student Portal adalah hasil skripsi Herfan Heryandi \cite{herfan:15:portal} dan kontributor lainnya. Saat ini IF Student Portal telah mendukung kurikulum 2018 berkat kontribusi skripsi Andrianto Sugiarto \cite{andrianto:18:portalsiam}. Mahasiswa Teknik Informatika dapat mengakses IF Student Portal melalui URL \url{https://ifstudentportal.herokuapp.com/}. Untuk mengakses Portal Akademik Mahasiswa, mahasiswa harus \textit{login} menggunakan akun email \textit{student}. 

SIAModels merupakan \textit{library} Java yang merepresentasikan Sistem Informasi Akademik Teknik Informatika UNPAR \cite{siamodels}. Saat ini SIAModels mendukung kurikulum 2018. 


\section{Analisis sistem usulan}
Analisis
\subsection{Persiapan}
Sebelum dan sambil proses pengembangan aplikasi \textit{android} IF Student Portal berjalan, ada beberapa pekerjaan yang dilakukan sebagai berikut :
\begin{itemize}
    \item Perawatan situs IF Student Portal\\
    IF Student Portal beberapa kali mengalami gangguan sehingga tidak bisa digunakan sebagaimana mestinya, sehingga dilakukan perbaikan-perbaikan yang diperlukan. Perubahan dapat dilihat di \url{https:www.github.com/ftisunpar/ifstudentportal/}.
    \item Perawatan SIA Models\\
    SIA Models versi 3.1.0 belum cocok untuk digunakan di platform \textit{android} sehingga perlu sedikit modifikasi agar bisa digunakan untuk pembangunan \textit{android} IF Student Portal. Perubahan dapat dilihat di \url{https:www.github.com/pascalalfadian/siamodels/}.
    
\end{itemize}

\subsection{Analisis Kebutuhan \textit{Android} IF Student Portal}
Aplikasi \textit{android} IF Student Portal akan memiliki fitur-fitur yang sudah ada di situs IF Student Portal. Fitur-fitur tersebut akan diimplementasikan dan diintegrasikan dengan fitur-fitur \textit{android} jika memungkinkan. Untuk mengetahui fitur apa lagi yang dibutuhkan...

Fitur-fitur baru yang akan diimplementasikan harus memenuhi kriteria berikut :
\begin{itemize}
    \item Data yang diperlukan tersedia di Student Portal UNPAR.
    \item Data hasil olahan tidak tersedia  di Student Portal UNPAR.
    \item Fitur mendukung fungsi Student Portal sebagai sumber informasi akademik.
\end{itemize}
\subsection{Analisis Use Case}
\subsection{Analisis Kelas}