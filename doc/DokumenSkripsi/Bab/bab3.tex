\chapter{Analisis}
\label{chap:analisis}

Pada bab ini akan dijelaskan analisis sistem yang sudah berjalan dan sistem yang akan dibangun. Analisis yang akan dibahas meliputi analisis \textit{use case}, analisis kelas, analisis kebutuhan fungsional dan non fungsional.

\section{Analisis Sistem Kini}
%Analisis student portal unpar dan if student portal
% 

IF Student Portal adalah situs yang diperuntukan bagi mahasiswa IF UNPAR\cite{ifstudentportalunpar}. IF Student Portal dibuat dengan tujuan menjadi pengembangan lebih lanjut dari Student Portal UNPAR pada masanya. IF Student Portal adalah hasil skripsi Herfan Heryandi \cite{herfan:15:portal} dan kontributor lainnya. Saat ini IF Student Portal telah mendukung kurikulum 2018 berkat kontribusi skripsi Andrianto Sugiarto \cite{andrianto:18:portalsiam}\footnote{Mahasiswa Teknik Informatika dapat mengakses IF Student Portal melalui URL \url{https://ifstudentportal.herokuapp.com/}}. Untuk mengakses Portal Akademik Mahasiswa, mahasiswa harus \textit{login} menggunakan akun email \textit{student}. 

SIA Models merupakan \textit{library} Java yang merepresentasikan Sistem Informasi Akademik Teknik Informatika UNPAR \cite{siamodels}. Saat ini SIAModels mendukung kurikulum 2018. 

\subsection{Persiapan}
Sebelum dan selama proses pengembangan aplikasi \textit{android} IF Student Portal berjalan, ada beberapa pekerjaan yang dilakukan sebagai berikut :
\begin{itemize}
    \item Perawatan situs IF Student Portal\\
    IF Student Portal beberapa kali mengalami gangguan sehingga tidak bisa digunakan sebagaimana mestinya, sehingga dilakukan perbaikan-perbaikan yang diperlukan. Perubahan yang dilakukan adalah sebagai berikut \footnote{Kode dapat dilihat di \url{https:www.github.com/ftisunpar/ifstudentportal/}.} :
    \begin{itemize}
        \item Memperbaiki \textit{bug} alamat foto profil dari Student Portal UNPAR.
        \item Memperbaiki \textit{bug} mengambil kode semester ddari Student Portal UNPAR dengan memindahkan halaman sumber kode semester dari halaman frs ke halaman nilai.
        \item Merubah penggunaan SIA Models dari \textit{submodule} ke \textit{Maven}.
    \end{itemize}
    \item Perawatan SIA Models\\
    SIA Models versi 3.1.0 belum cocok untuk digunakan di platform \textit{android} sehingga perlu sedikit modifikasi agar bisa digunakan untuk pembangunan \textit{android} IF Student Portal. Perubahan yang dilakukan adalah sebagai berikut\footnote{Kode dapat dilihat di \url{https:www.github.com/pascalalfadian/siamodels/}.} :
    \begin{itemize}
        \item Merubah nilai kembalian dari \textit{method getPhotoImage} menjadi \textit{byte}.
    \end{itemize}
\end{itemize}


\section{Analisis Sistem Usulan}
Analisis

\subsection{Analisis Kebutuhan \textit{Android} IF Student Portal}
\label{kebutuhan}
Aplikasi \textit{android} IF Student Portal akan memiliki fitur-fitur yang sudah ada di situs IF Student Portal. Fitur-fitur tersebut akan diimplementasikan dan diintegrasikan dengan fitur-fitur \textit{android} jika memungkinkan. Untuk mengetahui fitur apa lagi yang dibutuhkan, dibuat survey untuk meminta pendapat kepada mahasiswa Teknik Informatika angkatan 2016-2020. Dari hasil survey, diperoleh fitur yang diinginkan oleh mahasiswa adalah sebagai berikut :
\begin{itemize}
    \item Jadwal mata kuliah yang terintegrasi dengan kalender sistem.
    \item Notifikasi yang berisi informasi seputar perkuliahan.
    \item Simulasi sederhana untuk masa FRS (pengecekan jadwal bentrok, jumlah sks, prasyarat).
    \item Tempat melihat pengumuman yang bisa di urutkan dan rapi.
    \item Tempat melihat jadwal akademik di selain jadwal kuliah mahasiswa (jam kerja Tata Usaha, jadwal tes TOEFL). 
    \item Rincian mata kuliah (informasi presentase lulus, materi kuliah, tingkat kesulitan, dll)
    \item Ringkasan data akademik dan syarat kelulusan
\end{itemize}

Fitur-fitur baru yang akan diimplementasikan akan memenuhi kriteria berikut :
\begin{itemize}
    \item Data yang diperlukan tersedia di Student Portal UNPAR.
    \item Data hasil olahan tidak tersedia  di Student Portal UNPAR.
    \item Fitur mendukung fungsi Student Portal sebagai sumber informasi akademik.
    \item Bisa diintegrasikan dengan fitur android.
\end{itemize}

Hasil analisis fitur-fitur yang diinginkan berdasarkan kriteria di atas dan batas waktu pembangunan aplikasi dapat dilihat pada Tabel \ref{tab:3_hasil_fitur}.
\begin{table}[H]
	\centering
		\caption{Tabel Hasil Analisis Kebutuhan Informatika Student Portal}
    \begin{tabular}{|p{4.5cm}|p{2.5cm}|p{8cm}|}
		\hline
		Fitur & Dibuat/Tidak dibuat & Alasan\\
		\hline
		Simulasi FRS sederhana                      & Dibuat (mungkin)       & Data dapat diambil dari Portal Akademik Mahasiswa dan aturan prasyarat mata kuliah Program Studi Teknik Informatika sudah tersedia di SIA Models                   \\
		\hline
    Tempat melihat pengumuman yang bisa di urutkan dan rapi.                               & Dibuat (mungkin)      & Data dapat diambil dari student portal UNPAR                       \\
		\hline
    Tempat melihat jadwal akademik di selain jadwal kuliah mahasiswa (jam kerja Tata Usaha, jadwal tes TOEFL).    & Tidak Dibuat       &  Data tidak tersedia di student portal UNPAR \\
		\hline
    Jadwal kuliah yang terintegrasi dengan kalender sistem                       & Dibuat       & Data tersedia di student portal UNPAR dan terintegrasi dengan fitur android                                                     \\
		\hline
    Rincian mata kuliah                               & Tidak dibuat & Data tidak bisa diperoleh dari Portal Akademik Mahasiwa                                               \\
		\hline
    Notifikasi yang berisi informasi seputar perkuliahan                                    & Dibuat & Data tersedia di student portal UNPAR dan terintegrasi dengan fitur android                                                                        \\
	    \hline
	    Ringkasan data akademik dan syarat kelulusan                                  & Dibuat & Data tersedia di student portal UNPAR                                                                       \\
	    \hline
		
		\end{tabular}
	\label{tab:3_hasil_fitur}
\end{table}
\subsection{Analisis Use Case}
Diagram \textit{use case} hanya akan memiliki 1 aktor yaitu mahasiswa Teknik Informatika UNPAR. Diagram \textit{use case} dapat dilihat di gambar \ref{}. 

Berdasarkan hasil analisis kebutuhan pada subbab \ref{kebutuhan}, dari x fitur yang akan dibuat, terdapat y buah \textit{use case} yaitu :
\begin{enumerate}
    \item \textbf{Melihat Ringkasan Data Akademik}
		\begin{itemize}
			\item Nama: Melihat ringkasan data akademik
			\item Aktor: Mahasiswa
			\item Deskripsi: melihat data mengenai mata kuliah apa saja yang sudah lulus beserta jenis mata kuliahnya(wajib atau pilihan), sisa SKS untuk mencapai kelulusan, dan mata kuliah wajib yang belum ditempuh. 
			\item Kondisi awal: Mahasiswa telah \textit{login}
			\item Kondisi akhir: Halaman ditampilkan dan berisi ringkasan data akademik
			\item Skenario utama: \\ \\
			\begin{tabular}{|p{0.5cm} |p{6cm}| p{6cm}|}
						\hline
							No 	& Aksi Aktor & Reaksi Sistem \\ \hline
							1 	& Mahasiswa memilih menu ringkasan data akademik. 	&	Sistem meringkas data kademik mahasiswa kemudian menampilkan halaman ringkasan data akademik \\ \hline 
						\end{tabular} 
			%\item Eksepsi: Mahasiswa sedang menempuh semester 1
		\end{itemize}
\end{enumerate}



\subsection{Analisis Kelas}





% Student Portal UNPAR adalah situs yang diperuntukan bagi mahasiswa UNPAR untuk mengakses informasi yang berhubungan dengan studi dan mahasiswa pemilik akun\cite{BTI:2012}. Mahasiswa dapat mengakses Student Portal UNPAR melalui URL \url{https://studentportal.unpar.ac.id/}. Untuk mengakses Portal Akademik Mahasiswa, mahasiswa harus \textit{login} menggunakan akun email \textit{student}. 