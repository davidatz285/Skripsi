\chapter{Perancangan}
\label{chap:perancangan}

Pada bab ini akan dijelaskan mengenai perancangan aplikasi yang akan dibangun meliputi diagram kelas rinci beserta deskripsi dan fungsinya, dan perancangan antarmuka.

\section{Diagram Kelas}
Diagram kelas dapat dilihat di gambar \ref{fig:4_final_class_diagram}. Penjelasan mengenai kelas dan \textit{method} yang ada di dalamnya adalah sebagai berikut :

\begin{enumerate}
		\item Scrapper\\
		Kelas ini berfungsi untuk mengambil data dari Portal Akademik Mahasiswa untuk diolah dan ditampilkan. Atribut yang dimiliki kelas ini adalah sebagai berikut :
		\begin{itemize}
			\item \textbf{String phpSessId:} menyimpan \textit{php session id} yang dibuat saat terjadi koneksi ke Portal Akademik Mahasiswa.
			\item \textbf{String ciSession:} menyimpan \textit{Code Igniter} session yang dibuat saat terjadi koneksi ke Portal Akademik Mahasiswa.
            \item \textbf{Mahasiswa mahasiswa:} merepresentasikan mahasiswa yang akan diolah dan ditampilkan datanya.
            \item \textbf{Context context:} menyimpan \textit{context} tujuan pengiriman data setelah diambil dan diolah dari Portal Akademik Mahasiswa.
            \item \textbf{HomeManager homeManager:} menyimpan objek yang mengatur halaman utama aplikasi.
            \item \textbf{LoginManager loginManager:} menyimpan objek yang mengatur halaman \textit{login} aplikasi.
            \item \textbf{JadwalManager jadwalManager:} menyimpan objek yang mengatur halaman jadwal mahasiswa.
            \item \textbf{ProgressDialog dialog:} menyimpan objek yang mengatur \textit{progress dialog}.
            \item \textbf{final String BASE_URL:} menyimpan \textit{url} ke Portal Akademik Mahasiswa.
            \item \textbf{final String LOGIN_URL:} menyimpan \textit{url} ke halaman \textit{login} Portal Akademik Mahasiswa.
            \item \textbf{final String SSO_URL:} menyimpan \textit{url} ke halaman login \textit{Single Sign On} UNPAR.
            \item \textbf{final String JADWAL_URL:} menyimpan \textit{url} ke halaman jadwal Portal Akademik Mahasiswa.
            \item \textbf{final String NILAI_URL:} menyimpan \textit{url} ke halaman nilai Portal Akademik Mahasiswa.
            \item \textbf{final String TOEFL_URL:} menyimpan \textit{url} ke halaman nilai TOEFL Portal Akademik Mahasiswa.
            \item \textbf{final String HOME_URL:} menyimpan \textit{url} ke halaman utama Portal Akademik Mahasiswa.

		\end{itemize}
		
		\textit{Method} yang dimiliki kelas ini adalah sebagai berikut :
		\begin{itemize}
			
			\item \textbf{public Scrapper(Context context, HomeManager homeManager)}\\
				Berfungsi sebagai \textit{constructor} kelas Scrapper.\\
				\textbf{Kembalian:} objek kelas scrapper untuk digunakan di halaman utama setelah pengguna \textit{login}.
            \item \textbf{public Scrapper(Context context, LoginManager loginManager)}\\
				Berfungsi sebagai \textit{constructor} kelas Scrapper.\\
				\textbf{Kembalian:} objek kelas scrapper untuk digunakan di halaman \textit{login} saat pengguna akan \textit{login}.	
			\item \textbf{public Scrapper(Context context, JadwalManager jadwalManager)}\\
				Berfungsi sebagai \textit{constructor} kelas Scrapper.\\
				\textbf{Kembalian:} objek kelas scrapper untuk digunakan di halaman jadwal mahasiswa.
			\item \textbf{public void login(String email, String password)}\\
			\textbf{Parameter:}
				\begin{itemize}
					\item \textbf{email:} \textit{email} mahasiswa yang dipakai \textit{login}.
					\item \textbf{password:} \textit{password} mahasiswa yang dipakai \textit{login}.
				\end{itemize}
				Berfungsi untuk \textit{login} ke Portal Akademik Mahasiswa agar bisa mengambil data mahasiswa.\\
				\textbf{Kembalian:} tidak ada.
		    \item \textbf{public void getMahasiswaInfo(String phpSessId, String npm)}\\
			\textbf{Parameter:}
				\begin{itemize}
					\item \textbf{phpSessId:} \textit{php session id} dari koneksi ke Portal Akademik Mahasiswa.
					\item \textbf{npm:} nomor pokok mahasiswa yang akan diambil datanya.
				\end{itemize}
				Berfungsi untuk mengambil data mahasiswa untuk ditampilkan di halaman utama.\\
				\textbf{Kembalian:} tidak ada.
			\item \textbf{public void getListJadwal(String phpSessId, String npm)}\\
			\textbf{Parameter:}
				\begin{itemize}
					\item \textbf{phpSessId:} \textit{php session id} dari koneksi ke Portal Akademik Mahasiswa.
				\end{itemize}
				Berfungsi untuk mengambil data jadwal mahasiswa untuk ditampilkan di halaman jadwal.\\
				\textbf{Kembalian:} tidak ada.
		\end{itemize}
		
		Terdapat 3 \textit{private class} di dalam kelas Scrapper ini. Kelas ini merupakan perluasan dari kelas \textit{AsyncTask}. Hal ini diperlukan karena untuk mengakses internet tidak bisa dilakukan di \textit{thread} utama \textit{Android}. Semua kelas ini memiliki \textit{method-method} yang harus diimplementasi dari kelas \textit{AsyncTask} yaitu \textbf{onPreExecute, doInBackground,} dan \textbf{onPostExecute}. Berikut penjelasan dari \textit{method-method} yang ada di dalam beberapa \textit{private class} di dalam kelas Scrapper.
		
		\begin{enumerate}
		    \item Login\\
		    Kelas ini digunakan untuk \textit{login} ke Portal Akademik Mahasiswa untuk mengambil data. \textit{Method} \textbf{onPreExecute} di kelas ini berfungsi menampilkan \textit{progress dialog} dengan teks keterangan sedang mencoba \textit{login}. \textit{Method} \textbf{doInBackground} di kelas ini melakukan \textit{login} ke Portal Akademik Mahasiswa dengan \textit{email} dan kata sandi mahasiswa. \textit{Method} \textbf{onPostExecute} di kelas ini berfungsi memindahkan aplikasi dari halaman \textit{login} ke halaman utama setelah berhasil \textit{login}. 
		    \item GetMahasiswaInfo\\
		    Kelas ini digunakan untuk mengambil data mahasiswa dari Portal Akademik Mahasiswa untuk diolah dan ditampilkan kemudian. Tidak ada \textit{method} \textbf{onPreExecute} di kelas ini. \textit{Method} \textbf{doInBackground} di kelas ini melakukan pengambilan data nama, npm, dan foto profil mahasiswa dari Portal Akademik Mahasiswa. \textit{Method} \textbf{onPostExecute} di kelas ini berfungsi mengirimkan data ke kelas lain yang berfungsi menampilkan data tersebut di halaman utama. 
		    \item RequestJadwal\\
		    Kelas ini digunakan untuk mengambil data jadwal mahasiswa dari Portal Akademik Mahasiswa untuk diolah dan ditampilkan kemudian. Tidak ada \textit{method} \textbf{onPreExecute} di kelas ini. \textit{Method} \textbf{doInBackground} di kelas ini melakukan pengambilan data mahasiswa seperti nama mata kuliah, kode mata kuliah, hari dan jam kelas, dan nama dosen dan  dari Portal Akademik Mahasiswa. \textit{Method} \textbf{onPostExecute} di kelas ini berfungsi mengirimkan data ke kelas lain yang berfungsi menampilkan data tersebut di halaman jadwal. 
		\end{enumerate}
\end{enumerate}

\begin{figure}[ht]
			\centering
			\includegraphics[scale=0.45]{Gambar/class-diagram-final}
			\caption{Diagram Kelas} 
			\label{fig:4_final_class_diagram}
		\end{figure}

\section{Perancangan Antarmuka}
