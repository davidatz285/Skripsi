\documentclass[a4paper,twoside]{article}
\usepackage[T1]{fontenc}
\usepackage[bahasa]{babel}
\usepackage{graphicx}
\usepackage{graphics}
\usepackage{float}
\usepackage[cm]{fullpage}
\pagestyle{myheadings}
\usepackage{etoolbox}
\usepackage{setspace} 
\usepackage{lipsum} 
\setlength{\headsep}{30pt}
\usepackage[inner=2cm,outer=2.5cm,top=2.5cm,bottom=2cm]{geometry} %margin
% \pagestyle{empty}

\makeatletter
\renewcommand{\@maketitle} {\begin{center} {\LARGE \textbf{ \textsc{\@title}} \par} \bigskip {\large \textbf{\textsc{\@author}} }\end{center} }
\renewcommand{\thispagestyle}[1]{}
\markright{\textbf{\textsc{AIF401/AIF402 \textemdash Rencana Kerja Skripsi \textemdash Sem. Ganjil 2020/2021}}}

\newcommand{\HRule}{\rule{\linewidth}{0.4mm}}
\renewcommand{\baselinestretch}{1}
\setlength{\parindent}{0 pt}
\setlength{\parskip}{6 pt}

\onehalfspacing
 
\begin{document}

\title{\@judultopik}
\author{\nama \textendash \@npm} 

%tulis nama dan NPM anda di sini:
\newcommand{\nama}{David Christopher}
\newcommand{\@npm}{2017730015}
\newcommand{\@judultopik}{Pembangunan Aplikasi \textit{Android} IF Student Portal } % Judul/topik anda
\newcommand{\jumpemb}{1} % Jumlah pembimbing, 1 atau 2
\newcommand{\tanggal}{08/10/2020}

% Dokumen hasil template ini harus dicetak bolak-balik !!!!

\maketitle

\pagenumbering{arabic}

\section{Deskripsi}
Student Portal UNPAR adalah sistem informasi berbasis \textit{web} yang digunakan oleh mahasiswa UNPAR. IF Student Portal adalah sistem informasi berbasis \textit{web} khusus mahasiswa Teknik Informatika UNPAR hasil pengembangan yang lebih lanjut dari Student Portal UNPAR sebagai skripsi milik Herfan Heryandi. Skripsi ini dibuat untuk membawa IF Student Portal ke platform \textit{android} sebagai aplikasi \textit{native android} IF Student Portal. Aplikasi \textit{android} IF Student Portal akan memiliki arsitektur \textit{MVP(Model-View-Presenter}, dibangun dengan bantuan \textit{library} jsoup, dan mengikuti kaidah \textit{android design} untuk tampilannya.  

\section{Rumusan Masalah}
Rumusan masalah yang akan dibahas di skripsi ini adalah sebagai berikut :
\begin{itemize}
    \item Fitur apa saja yang akan tersedia di aplikasi \textit{android} IF Student Portal?
    \item Bagaimana membangun aplikasi \textit{android} IF Student Portal yang memiliki fitur-fitur situs IF  Student Portal?
    \item Bagaimana membangun tampilan aplikasi \textit{android} IF Student Portal senada situs IF Student Portal dengan mengikuti kaidah \textit{android design}?
    
\end{itemize}

\section{Tujuan}
Tujuan yang ingin dicapai dari penulisan skripsi ini sebagai berikut :
\begin{itemize}
    \item Mendefinisikan fitur apa saja yang tersedia di IF Student Portal.
    \item Membangun aplikasi \textit{native android} IF Student Portal yang memiliki fitur-fitur dari situs IF Student Portal.
    \item Membangun tampilan aplikasi \textit{android} IF Student Portal senada situs IF Student Portal dengan mengikuti kaidah \textit{android design}.
    
\end{itemize}

\section{Deskripsi Perangkat Lunak}

Perangkat lunak akhir yang akan dibuat memiliki fitur minimal sebagai berikut:
\begin{itemize}
	\item Pengguna dapat melihat jadwal perkuliahan yang diikuti.
	\item Pengguna dapat melihat data diri.
	\item Perangkat lunak memiliki fitur-fitur yang dimiliki situs IF Student Portal.
	\item Perangkat lunak memiliki mode luring.
	\item Perangkat lunak mendukung kurikulum 2018.
	\item Perangkat lunak memiliki tampilan yang senada situs IF Student Portal dan mengikuti kaidah \textit{android design}.
	\item Fitur lain yang dirasa dibutuhkan setelah analisis kebutuhan akan ditambahkan kemudian.
\end{itemize}

\section{Detail Pengerjaan Skripsi}

Bagian-bagian pekerjaan skripsi ini adalah sebagai berikut :
	\begin{enumerate}
		\item Melakukan studi mengenai \textit{library} jsoup.
		\item Melakukan studi mengenai \textit{android design}.
		\item Menganalisis IF Student Portal, Student Portal UNPAR, dan kebutuhan aplikasi.
		\item Mendesain kelas aplikasi.
		\item Membangun aplikasi \textit{android} IF Student portal dengan \textit{library} jsoup.
		\item Mengimplementasikan tampilan aplikasi sesuai rekomendasi \textit{android design} dan senada terhadap situs IF Student Portal.
		\item Melakukan pengujian dan eksperimen.
		\item Menulis dokumen skripsi.
	\end{enumerate}

\section{Rencana Kerja}
Rincian capaian yang direncanakan di Skripsi 1 adalah sebagai berikut:
\begin{enumerate}
\item Mempelajari \textit{library} jsoup.
\item Menganalisis situs IF Student Portal, Student Portal UNPAR, dan kebutuhan aplikasi.
\item Merancang model aplikasi \textit{android} IF Student Portal.
\item Mengimplementasi \textit{android} IF Student Portal dengan \textit{library} jsoup.
\item Menulis dokumen skripsi (Bab 1-3).
\end{enumerate}

Sedangkan yang akan diselesaikan di Skripsi 2 adalah sebagai berikut:
\begin{enumerate}
\item Mempelajari \textit{android design}.
\item Membuat tampilan aplikasi senada dengan situs IF Student Portal dengan mengikuti kaidah \textit{android design}.
\item Melakukan pengujian dan eksperimen terhadap aplikasi yang dibuat.
\item Menulis dokumen skripsi (Bab 4-6).
\end{enumerate}

\vspace{1cm}
\centering Bandung, \tanggal\\
\vspace{2cm} \nama \\ 
\vspace{1cm}

Menyetujui, \\
\ifdefstring{\jumpemb}{2}{
\vspace{1.5cm}
\begin{centering} Menyetujui,\\ \end{centering} \vspace{0.75cm}
\begin{minipage}[b]{0.45\linewidth}
% \centering Bandung, \makebox[0.5cm]{\hrulefill}/\makebox[0.5cm]{\hrulefill}/2013 \\
\vspace{2cm} Nama: \makebox[3cm]{\hrulefill}\\ Pembimbing Utama
\end{minipage} \hspace{0.5cm}
\begin{minipage}[b]{0.45\linewidth}
% \centering Bandung, \makebox[0.5cm]{\hrulefill}/\makebox[0.5cm]{\hrulefill}/2013\\
\vspace{2cm} Nama: \makebox[3cm]{\hrulefill}\\ Pembimbing Pendamping
\end{minipage}
\vspace{0.5cm}
}{
% \centering Bandung, \makebox[0.5cm]{\hrulefill}/\makebox[0.5cm]{\hrulefill}/2013\\
\vspace{2cm} Nama: \makebox[3cm]{\hrulefill}\\ Pembimbing Tunggal
}
\end{document}

