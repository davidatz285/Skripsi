\documentclass[a4paper,twoside]{article}
\usepackage[T1]{fontenc}
\usepackage[bahasa]{babel}
\usepackage{graphicx}
\usepackage{graphics}
\usepackage{float}
\usepackage[cm]{fullpage}
\pagestyle{myheadings}
\usepackage{etoolbox}
\usepackage{setspace} 
\usepackage{lipsum} 
\setlength{\headsep}{30pt}
\usepackage[inner=2cm,outer=2.5cm,top=2.5cm,bottom=2cm]{geometry} %margin
% \pagestyle{empty}

\makeatletter
\renewcommand{\@maketitle} {\begin{center} {\LARGE \textbf{ \textsc{\@title}} \par} \bigskip {\large \textbf{\textsc{\@author}} }\end{center} }
\renewcommand{\thispagestyle}[1]{}
\markright{\textbf{\textsc{AIF401/AIF402 \textemdash Rencana Kerja Skripsi \textemdash Sem. Ganjil 2020/2021}}}

\newcommand{\HRule}{\rule{\linewidth}{0.4mm}}
\renewcommand{\baselinestretch}{1}
\setlength{\parindent}{0 pt}
\setlength{\parskip}{6 pt}

\onehalfspacing
 
\begin{document}

\title{\@judultopik}
\author{\nama \textendash \@npm} 

%tulis nama dan NPM anda di sini:
\newcommand{\nama}{David Christopher}
\newcommand{\@npm}{2017730015}
\newcommand{\@judultopik}{IF Student Portal : Pemanfaatan \textit{Web Scraping} untuk Pembangunan Aplikasi \textit{Android}} % Judul/topik anda
\newcommand{\jumpemb}{1} % Jumlah pembimbing, 1 atau 2
\newcommand{\tanggal}{08/10/2020}

% Dokumen hasil template ini harus dicetak bolak-balik !!!!

\maketitle

\pagenumbering{arabic}

\section{Deskripsi}
Student Portal UNPAR adalah sistem informasi berbasis \textit{web} yang digunakan oleh mahasiswa UNPAR. IF Student Portal adalah sistem informasi berbasis \textit{web} khusus mahasiswa Teknik Informatika UNPAR hasil pengembangan yang lebih lanjut dari Student Portal UNPAR. Skripsi ini dibuat untuk membawa IF Student Portal ke platform \textit{android} sebagai aplikasi \textit{native android} IF Student Portal dengan menggunakan \textit{library} jsoup untuk mengimplementasikan web scraping. \textit{Web scraping} adalah teknik untuk mendapatkan informasi tertentu dengan cara yang semi terstruktur dari sebuah \textit{website}. \textit{Web scraping} bisa digunakan ketika \textit{website} yang memiliki informasi informasi yang kita cari tidak menyediakan \textit{API (Application Programming Interface)}. 

\section{Rumusan Masalah}
Rumusan masalah yang akan dibahas di skripsi ini adalah sebagai berikut :
\begin{itemize}
    \item Fitur apa saja yang akan tersedia di aplikasi \textit{android} IF Student Portal?
    \item Bagaimana cara mengimplementasikan \textit{web scraping} dengan \textit{library} jsoup?
    \item Bagaimana membangun aplikasi \textit{android} IF Student Portal?
    
\end{itemize}

\section{Tujuan}
Tujuan yang ingin dicapai dari penulisan skripsi ini sebagai berikut :
\begin{itemize}
    \item Mengetahui fitur apa saja yang akan tersedia di IF Student Portal.
    \item Mengimplementasikan \textit{web scraping} menggunakan \textit{library} jsoup.
    \item Membangun aplikasi \textit{native android} IF Student Portal.
    
\end{itemize}

\section{Deskripsi Perangkat Lunak}

Perangkat lunak akhir yang akan dibuat memiliki fitur minimal sebagai berikut:
\begin{itemize}
	\item Pengguna dapat melihat jadwal perkuliahan yang diikuti.
	\item Pengguna dapat melihat data diri.
	\item Perangkat lunak dapat diakses dimana saja dengan koneksi internet.
	\item Perangkat lunak memiliki \textit{offline mode}.
	\item Perangkat lunak mendukung kurikulum 2018.
	\item Perangkat lunak memiliki tampilan yang estetik, responsif, dan senada dengan \textit{website} IF Student Portal.
\end{itemize}

\section{Detail Pengerjaan Skripsi}
Tuliskan bagian-bagian pengerjaan skripsi secara detail. Bagian pekerjaan tersebut mencakup awal hingga akhir skripsi, termasuk di dalamnya pengerjaan dokumentasi skripsi, pengujian, survei, dll.

Bagian-bagian pekerjaan skripsi ini adalah sebagai berikut :
	\begin{enumerate}
		\item Melakukan studi mengenai \textit{library} jsoup dan teknik \textit{web scraping}.
		\item Melakukan studi mengenai \textit{android design}.
		\item Menganalisis IF Student Portal dan Student Portal UNPAR.
		\item Mendesain kelas aplikasi.
		\item Mengimplementasikan \textit{web scraping} menggunakan \textit{library} jsoup.
		\item Mengimplementasikan tampilan aplikasi sesuai rekomendasi \textit{android design} dan konsisten terhadap \textit{website} IF Student Portal.
		\item Melakukan pengujian dan eksperimen.
		\item Menulis dokumen skripsi.
	\end{enumerate}

\section{Rencana Kerja}
Rincian capaian yang direncanakan di Skripsi 1 adalah sebagai berikut:
\begin{enumerate}
\item Mempelajari teknik \textit{web scraping} dengan \textit{library} JSOUP.
\item Menganalisis IF Student Portal dan Student Portal UNPAR.
\item Merancang model aplikasi \textit{android} IF Student Portal.
\item Mengimplementasikan \textit{web scraping} pada aplikasi \textit{android} dengan \textit{library} jsoup
\item Menulis dokumen skripsi (BAB 4-6).
\end{enumerate}

Sedangkan yang akan diselesaikan di Skripsi 2 adalah sebagai berikut:
\begin{enumerate}
\item Mempelajari \textit{android design}.
\item Membuat tampilan aplikasi yang intuitif, estetik, responsif, dan konsisten terhadap \textit{website} IF Student Portal.
\item Melakukan pengujian dan eksperimen terhadap aplikasi yang dibuat.
\item Menulis dokumen skripsi (BAB 4-6).
\end{enumerate}

% \begin{center}
%   \begin{tabular}{ | c | c | c | c | l |}
%     \hline
%     1*  & 2*(\%) & 3*(\%) & 4*(\%) &5*\\ \hline \hline
%     1   & 10  & 10  &  &  Mempelajari teknik \textit{web scraping} dan \textit{library} jsoup\\ \hline
%     2   & 10 & 5  & 5  & Mempelajari \textit{android design} \\ \hline
%     3   & 10  & 10  &  & Menganalisis IF Student Portal dan Student Portal UNPAR\\ \hline
%     4   & 15  & 15  &  & Merancang struktur program\\ \hline
%     5   & 15  & 5  & 15 & Membangun aplikasi dengan teknik \textit{web scraping}\\ \hline
%     6   & 10  & 0  & 10 & Implementasi tampilan \\ \hline
%     7   & 10  & 0  & 10 & Melakukan pengujian dan eksperimen\\  \hline
%     8   & 20  & 5  & 15 &  Menulis dokumen skripsi\\ \hline
%     Total  & 100  & 50  & 50 &  \\ \hline
%                           \end{tabular}
% \end{center}

% Keterangan (*)\\
% 1 : Bagian pengerjaan Skripsi (nomor disesuaikan dengan detail pengerjaan di bagian 5)\\
% 2 : Persentase total \\
% 3 : Persentase yang akan diselesaikan di Skripsi 1 \\
% 4 : Persentase yang akan diselesaikan di Skripsi 2 \\
% 5 : Penjelasan singkat apa yang dilakukan di S1 (Skripsi 1) atau S2 (skripsi 2)

\vspace{1cm}
\centering Bandung, \tanggal\\
\vspace{2cm} \nama \\ 
\vspace{1cm}

Menyetujui, \\
\ifdefstring{\jumpemb}{2}{
\vspace{1.5cm}
\begin{centering} Menyetujui,\\ \end{centering} \vspace{0.75cm}
\begin{minipage}[b]{0.45\linewidth}
% \centering Bandung, \makebox[0.5cm]{\hrulefill}/\makebox[0.5cm]{\hrulefill}/2013 \\
\vspace{2cm} Nama: \makebox[3cm]{\hrulefill}\\ Pembimbing Utama
\end{minipage} \hspace{0.5cm}
\begin{minipage}[b]{0.45\linewidth}
% \centering Bandung, \makebox[0.5cm]{\hrulefill}/\makebox[0.5cm]{\hrulefill}/2013\\
\vspace{2cm} Nama: \makebox[3cm]{\hrulefill}\\ Pembimbing Pendamping
\end{minipage}
\vspace{0.5cm}
}{
% \centering Bandung, \makebox[0.5cm]{\hrulefill}/\makebox[0.5cm]{\hrulefill}/2013\\
\vspace{2cm} Nama: \makebox[3cm]{\hrulefill}\\ Pembimbing Tunggal
}
\end{document}

